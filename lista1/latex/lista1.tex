\documentclass{article}

\usepackage{polski}
\usepackage[letterpaper,top=2cm,bottom=2cm,left=3cm,right=3cm,marginparwidth=1.75cm]{geometry}
\usepackage{amsmath}
\usepackage{graphicx}
\usepackage{algpseudocode}
\usepackage{pgfplots}
\pgfplotsset{compat=1.18}
\usepackage[colorlinks=true, allcolors=blue]{hyperref}

\title{Obliczenia naukowe lista 1}
\author{Stanisław Tomkowiak}

\begin{document}
\maketitle



\section*{Zadanie 1}
\subsection*{Macheps}

Epsilon maszynowy czyli właśnie macheps jest to najmniejsza liczba taka, żę $1+macheps >1 $. Dzięki temu epsilonowi określamy precyzję arytmetyki. Błąd zaokrąglenia liczby wynosi dokładnie $\frac{1}{2}$ macheps.

Wyznaczenie wartości epsilonu maszynowego metodą iteracyjną polega na dodawaniu do liczby 1.0 w wybranej precyzji coraz mniejszych wartości, dopóki liczba jest większa od 1.0. Jako pierwszą liczbę ustalamy macheps=1.0 i w każdej iteracji dzielimy go przez 2. Pętla zatrzyma się na wartości macheps.


\begin{algorithm}
\caption{Obliczanie maszynowego epsilonu dla typu liczbowego \( T \):}
\begin{algorithmic}[1]
\Function{macheps}{T}
    \State \( \text{macheps} \gets T(1.0) \) \Comment{Inicjalizujemy macheps jako 1.0 dla typu \( T \)}
    \While{ \( T(1.0) + \frac{\text{macheps}}{T(2.0)} > T(1.0) \)}
        \State \( \text{macheps} \gets \frac{\text{macheps}}{T(2.0)} \) \Comment{Dzielimy macheps przez 2, dopóki sumowanie go z T(1.0) jest większe od 1.0 }
    \EndWhile
    \State \Return \text{macheps} \Comment{Zwracamy obliczoną wartość macheps}
\EndFunction
\end{algorithmic}
\end{algorithm}

Poniżej przedstawiam wyniki z dokładnością do trzech cyfr znaczących dla trzech różnych metod wyznaczenia epsilonu:
\begin{center}
  \begin{tabular}{|c|c|c|c|}
    \hline
            & Float16   & Float32    & Float64   \\ [0.5ex]
    \hline
    wyniki iteracyjne& 0.000977& $1.19 \cdot 10^{-7}$ & $2.22 \cdot 10^{-16}$ \\
    \hline
    wyniki uzyskane za pomocą funkcji eps() & 0.000977& $1.19 \cdot 10^{-7}$ & $2.22 \cdot 10^{-16}$ \\
    \hline
    wartości w float.h   & brak & $1.19 \cdot 10^{-7}$ & $2.22 \cdot 10^{-16}$ \\
    \hline
  \end{tabular}
\end{center}

Wyniki dla tych trzech metod są takie same. Znaczy to, że iteracyjny sposób wyznaczenia epsilonu maszynowego jest skuteczny.
\subsection*{Eta}
Eta jest to najmniejsza liczba większa od zera maszynowego. W zadaniu należało obliczyć iteracyjnie wartość liczby eta dla wszystkich typów zmiennopozycyjnych zgodnych ze standardem IEEE 754. Dodatkowo musieliśmy zbadać związek tej liczby z liczbą $MIN_{sub}$.

Wyznaczenie tych wartości jest analogiczne do wyznaczania macheps. Dzielimy liczbę 1.0 przez 2 do czasu aż jest ona większa od 0.0. Poniżej przedstawiam pseudokod:


\begin{algorithm}
\caption{Obliczanie liczby eta dla typu liczbowego \( T \)}
\begin{algorithmic}[1]
\Function{eta}{T}
    \State \( \text{eta} \gets T(1.0) \) \Comment{Inicjalizujemy eta jako 1.0 dla typu \( T \)}
    \While{ \(\frac{\text{eta}}{T(2.0)} > T(0.0) \)}
        \State \( \text{eta} \gets \frac{\text{eta}}{T(2.0)} \) \Comment{Dzielimy eta przez 2, dopóki jest większa od 0.0 }
    \EndWhile
    \State \Return \text{eta} \Comment{Zwracamy obliczoną wartość macheps}
\EndFunction
\end{algorithmic}
\end{algorithm}
    Poniżej przedstawiam wyniki dwóch różnych metod wyznaczenia eta:
\begin{center}
  \begin{tabular}{|c|c|c|c|}
    \hline
       & Float16 & Float32 & Float64  \\ [0.5ex]
    \hline
    wyniki iteracyjne & 6.0e-8  & 1.0e-45 & 5.0e-324 \\
    \hline
    wyniki uzyskane za pomocą funkcji nextfloat(0.0) & 6.0e-8  & 1.0e-45 & 5.0e-324 \\
    \hline
  \end{tabular}
\end{center}

Wyniki dla tych dwóch metod są takie same. Znaczy to, że iteracyjny sposób wyznaczenia liczby eta jest skuteczny.

Związek liczby eta z $MIN_{sub}$. $MIN_{sub}$ jest najmniejszą liczbą większą od 0.0 w formie nieznormalizowanej. Po obliczeniu liczby $MIN_{sub}$ dla badanych typów jest ona równa liczbie eta. Wyniki funkcji minfloat(Float32) i minfloat(Float64) prezentują się następująco:
\begin{itemize}
\item floatmin32: 1.1754944e-38
\item floatmin64: 2.2250738585072014e-308
\end{itemize}
Są one zatem wiele większe od iteracyjnej próby, wartości te są najmniejszymi wartościami większymi od 0.0 znormalizowanymi.
\subsection*{Max}
Największa wartość jaką można otrzymać w standardzie IEEE 754. Otrzymuję ją za pomocą algorytmu w którym mnożymy liczbę previous float(1.0) razy 2 tak długo aż liczba jest różna od wartości infinity. Poniżej przedstawiam pseudokod:

\begin{algorithm}
\caption{Obliczanie maksymalnej wartości reprezentowalnej dla typu liczbowego \( T \)}
\begin{algorithmic}[1]
\Function{max}{T}
    \State \( \text{max} \gets \text{prevfloat}(T(1.0)) \) \Comment{Inicjalizujemy max jako największą liczbę mniejszą niż 1.0 dla typu \( T \)}
    \While{ \textbf{not} \(\text{isinf}(\text{max} \times T(2.0))\) }
        \State \( \text{max} \gets \text{max} \times T(2.0) \) \Comment{Mnożymy max przez 2, dopóki nie osiągnie nieskończoności}
    \EndWhile
    \State \Return \text{max} \Comment{Zwracamy obliczoną maksymalną wartość}
\EndFunction
\end{algorithmic}
\end{algorithm}
Poniżej przedstawiam wyniki badań:

\begin{center}
  \begin{tabular}{|c|c|c|c|}
    \hline
       & Float16           & Float32              & Float64               \\ [0.5ex]
    \hline
    wyniki iteracyjne  & $6.55 \cdot 10^4$ & $3.40 \cdot 10^{38}$ & $1.80 \cdot 10^{308}$ \\
    \hline
    wynikiuzyskane za pomocą funkcji floatmax() & $6.55 \cdot 10^4$ & $3.40 \cdot 10^{38}$ & $1.80 \cdot 10^{308}$ \\
    \hline
    wartości z raportu  & brak danych   & $3.40 \cdot 10^{38}$ & $1.80 \cdot 10^{308}$ \\
    \hline
  \end{tabular}
\end{center}
Wyniki są takie same. Znaczy to, że iteracyjna forma wyznaczenia maksymalnej wartości jest prawidłowa(zgadza się z danymi z raportu oraz wynikami funkcji floatmax).




\section*{Zadanie 2}
Wzór wymyślony przez Kahana: $3(\frac{4}{3}-1)-1$ po obliczeniu zwraca wyniki zgodne z machepsem danego typu co do znaku. Można uznać je za poprawne ponieważ w standardzie IEEE 754 pomnożenie przez -1 to jedynie negacja pierwszego bitu. Jednak aby uznać wzór kahana za w pełni poprawny należałoby zapisać go w formie:
\[
\left| 3 \left( \frac{4}{3} - 1 \right) - 1 \right|
\]
\section*{Zadanie 3}
Sprawdzenie rozmieszczenia liczb w standardzie IEEE 754 arytmetyce Float64. Z powodów bardzo dużej liczby liczb sprawdzamy tylko mały wycinek tych liczb z przedziału. Dla zbadanych wartości wynika, że przedziale [1,2] liczby zmiennopozycyjne są rozmieszczone równomiernie. Krok tego przedziału jest równy $\delta = 2^{-52}$. Zatem każda liczba występująca w tym przedziale może być przedstawiona za pomocą wzoru $x=1+k\delta$ dla $k=1,2,...,2^{-52}-1$. Badając jednak inne przedstawione w zadaniu przedziału obserwujemy iż kroki są inne. Przedział [0.5,1] ma krok równy $\delta = 2^{-53}$. Przedział [2,4] ma krok równy $\delta = 2^{-51}$. 

W badanym standardzie w reprezentacji bitowej obserwujemy,że przedziały $[2^k, 2^{k+1}]$  różnią się tylko mantysą. W każdym takim przedziale znajduje się $2^t$ równo rozmieszczonych liczb. Widzimy zatem, że im mniejsza cecha w danym przedziale tym bliższe sobie liczby jesteśmy w stanie zaobserwować.
\section*{Zadanie 4}
Najmniejsza liczba w arytmetyce Float64 w przedziale (1,2) taka, że $x*\frac{1}{x}\neq1$ to 1.000000057228997. Wyznaczyć ją można stosując metody poznane już wcześniej. Mianowicie zaczynając od następnej liczby po 1.0 wykonywać to działanie zwiększając liczbę o wartość macheps do czasu aż wartość iloczynu będzie różna od 1.0. Warto także sprawdzić jaki wynik otrzymamy po wykonaniu tego działania z tą liczbą. Jest ono równe: 0.9999999999999999. Wnioski z przeprowadzenia tego badania są takie, że ograniczenia w arytmetyce zmiennopozycyjnej mogą doprowadzać do błędów w wynikach przez co mogą zostać złamane pewne własności matematyczne.
\section*{Zadanie 5}
Zadanie polegało na obliczeniu iloczynu skalarnego dwóch podanych wektorów: 
\[
\begin{array}{r l}
x &= [2.718281828, -3.141592654, 1.414213562, 0.5772156649, 0.3010299957] \\
y &= [1486.2497, 878366.9879, -22.37492, 4773714.647, 0.000185049].
\end{array}
\]
Uzyskać wynik mieliśmy na cztery różne sposoby dla arytmetyk Float32 i Float64. Poniżej przedstawiam wyniki tych eksperymentów:

\begin{center}
  \begin{tabular}{|c|c|c|}
    \hline
      & Float32    & Float64 \\ [0.5ex]
    \hline
    prawidłowa wartość  & -1.00657107000000\cdot 10^{-11}& -1.00657107000000\cdot 10^{-11}$ \\
    \hline
    "w przód"  & -0.4999443 & $1.0251881368296672 \cdot 10^{-10}$ \\
    \hline
    "w tył"    & -0.4543457 & $-1.5643308870494366 \cdot 10^{-10}$ \\
    \hline
    "malejąco" & -0.5       & 0.0 \\
    \hline
    "rosnąco"  & -0.5       & 0.0 \\
    \hline
  \end{tabular}
\end{center}
Żaden ze sposobów nie jest skuteczny. Szczególnie duże błędy pojawiają się w precyzji single. Wyniki przedstawione w tabeli nie pojawiły się bez powodu; tak duże błędy wynikają z tego, że wektory te są prawie prostopadłe do siebie. Powstają przez to duże błędy w obliczeniach. Dodatkowo, dzięki temu zadaniu widać, iż kolejność wykonywania działań w komputerze ma znaczenie. Na błędy wpływa nie tylko arytmetyka, ale także właśnie kolejność działań.
\section*{Zadanie 6}
Zadanie polegało na obliczeniu wartości funkcji:
\[
f(x) = \sqrt{x^2 + 1} - 1
\]
\[
g(x) = \frac{x^2}{\sqrt{x^2 + 1} + 1}
\]
Dla różnych wartości argumentu \( x = 8^{-1}, 8^{-2}, 8^{-3}, \dots \). Warto również zaznaczyć, że z matematycznego sensu te dwie funkcje są  sobie równe. Należało policzyć to w precyzji Float64.
Funkcja \(f(x)\) jest równa 0.0 dla \(x = 8^{-9}\), natomiast funkcja \(g(x)\) jest równa 0.0 dla \(x = 8^{-179}\). Wynika z tego, że wyniki funkcji \(g(x)\) są dużo bardziej wiarygodne. Funkcja ta nigdy dla argumentów większych od 0 nie powinna osiągać wartości 0.0. Taka rozbieżność wyników wynika z tego, że korzystając z funkcji \(f(x)\) odejmujemy od siebie bardzo bliskie liczby przez co otrzymujemy duży błąd podczas obliczeń. W sytuacji użycia funkcji \(g(x)\) usuwamy ten problem. Przekształcając funkcje aby uniknąć odejmowania bliskich sobie liczb pozwala zwiększyć wiarygodność wyników, oraz zmniejszyć błędy podczas obliczeń.
\section*{Zadanie 7}
W zadaniu mamy policzyć przybliżoną wartość pochodnej w punkcie $x_0=1$. Zostaje nam do tego zadania dostarczony wzór:
\[f'=\frac{f(x_0+h)-f(x_0)}{h}\]
gdzie $h=2^{-n}$ ($n=0,1,2,...,54$). Poniżej przedstawiam wyniki na wykresie:
\begin{center}
\begin{tikzpicture}
    \begin{axis}[
        xlabel={$n$},
        ylabel={Wartość przybliżonej pochodnej w punkcie x_0=1},
        grid=major,
        width=14cm,
        height=8cm,
        legend pos=north east,
        yticklabel style={
            /pgf/number format/fixed,
            /pgf/number format/precision=6
        }
    ]
    
    % Dane do wykresu
    \addplot[blue, thick] coordinates {
        (1, 1.8704413979316472)
        (2, 1.1077870952342974)
        (3, 0.6232412792975817)
        (4, 0.3704000662035192)
        (5, 0.24344307439754687)
        (6, 0.18009756330732785)
        (7, 0.1484913953710958)
        (8, 0.1327091142805159)
        (9, 0.1248236929407085)
        (10, 0.12088247681106168)
        (11, 0.11891225046883847)
        (12, 0.11792723373901026)
        (13, 0.11743474961076572)
        (14, 0.11718851362093119)
        (15, 0.11706539714577957)
        (16, 0.11700383928837255)
        (17, 0.11697306045971345)
        (18, 0.11695767106721178)
        (19, 0.11694997636368498)
        (20, 0.11694612901192158)
        (21, 0.1169442052487284)
        (22, 0.11694324295967817)
        (23, 0.11694276239722967)
        (24, 0.11694252118468285)
        (25, 0.116942398250103)
        (26, 0.11694233864545822)
        (27, 0.11694231629371643)
        (28, 0.11694228649139404)
        (29, 0.11694222688674927)
        (30, 0.11694216728210449)
        (31, 0.11694216728210449)
        (32, 0.11694192886352539)
        (33, 0.11694145202636719)
        (34, 0.11694145202636719)
        (35, 0.11693954467773438)
        (36, 0.116943359375)
        (37, 0.1169281005859375)
        (38, 0.116943359375)
        (39, 0.11688232421875)
        (40, 0.1168212890625)
        (41, 0.116943359375)
        (42, 0.11669921875)
        (43, 0.1162109375)
        (44, 0.1171875)
        (45, 0.11328125)
        (46, 0.109375)
        (47, 0.109375)
        (48, 0.09375)
        (49, 0.125)
        (50, 0.0)
        (51, 0.0)
        (52, -0.5)
        (53, 0.0)
        (54, 0.0)
    };
    \end{axis}
\end{tikzpicture}
\end{center}
 Dodatkowo przedstawiam wyniki pomiaru błędu względem prawidłowej wartości pochodnej tej funkcji: 
\begin{figure}[h!]
    \centering
    \begin{tikzpicture}[yshift=-1cm] % przesunięcie wykresu o 1 cm w dół
        \begin{semilogyaxis}[
            width=12cm, % szerokość wykresu
            height=8cm, % wysokość wykresu
            xlabel={$n$}, % etykieta osi X
            ylabel={Błąd}, % etykieta osi Y
            title={Błąd w zależności od $n$}, % tytuł wykresu
            grid=both, % włączenie siatki
            ymin=1e-12, % minimalna wartość na osi Y
            ymax=40, % maksymalna wartość na osi Y
            legend pos=north west,
        ]
        % Dane: (n, błąd)
        \addplot coordinates {
            (1, 1.753499116243109)
            (2, 0.9908448135457593)
            (3, 0.5062989976090435)
            (4, 0.253457784514981)
            (5, 0.1265007927090087)
            (6, 0.0631552816187897)
            (7, 0.03154911368255764)
            (8, 0.015766832591977753)
            (9, 0.007881411252170345)
            (10, 0.0039401951225235265)
            (11, 0.001969968780300313)
            (12, 0.0009849520504721099)
            (13, 0.0004924679222275685)
            (14, 0.0002462319323930373)
            (15, 0.00012311545724141837)
            (16, 6.155759983439424e-5)
            (17, 3.077877117529937e-5)
            (18, 1.5389378673624776e-5)
            (19, 7.694675146829866e-6)
            (20, 3.8473233834324105e-6)
            (21, 1.9235601902423127e-6)
            (22, 9.612711400208696e-7)
            (23, 4.807086915192826e-7)
            (24, 2.394961446938737e-7)
            (25, 1.1656156484463054e-7)
            (26, 5.6956920069239914e-8)
            (27, 3.460517827846843e-8)
            (28, 4.802855890773117e-9)
            (29, 5.480178888461751e-8)
            (30, 1.1440643366000813e-7)
            (31, 1.1440643366000813e-7)
            (32, 3.5282501276157063e-7)
            (33, 8.296621709646956e-7)
            (34, 8.296621709646956e-7)
            (35, 2.7370108037771956e-6)
            (36, 1.0776864618478044e-6)
            (37, 1.4181102600652196e-5)
            (38, 1.0776864618478044e-6)
            (39, 5.9957469788152196e-5)
            (40, 0.0001209926260381522)
            (41, 1.0776864618478044e-6)
            (42, 0.0002430629385381522)
            (43, 0.0007313441885381522)
            (44, 0.0002452183114618478)
            (45, 0.003661031688538152)
            (46, 0.007567281688538152)
            (47, 0.007567281688538152)
            (48, 0.023192281688538152)
            (49, 0.008057718311461848)
            (50, 0.11694228168853815)
            (51, 0.11694228168853815)
            (52, 0.6169422816885382)
            (53, 0.11694228168853815)
            (54, 0.11694228168853815)
        };
        \addlegendentry{Błąd }
        \end{semilogyaxis}
    \end{tikzpicture}
    \caption{Wykres błędu w zależności od liczby iteracji $n$.}
    \label{fig:blad_vs_n}
\end{figure}

Możemy zaobserwować, że najbliższe przybliżenie mamy dla wartości $n=28$. Dalsze zmniejszanie $h$ powoduje wzrost błędu. Wynika to z arytmetyki Float64 oraz odejmowania pojawiającego się w tym wzorze. Dodatkowo, po bliższym przyjrzeniu się wartościom $h+1$ widzimy, że dla $n=53$ $h+1=1$. Przez ograniczenia arytmetyki w pewnym momencie zaczynamy podczas odejmowania ucinać coraz to większe fragmenty $h$ przez co tracimy na dokładności przybliżenia. Wniosek wynikający z tego zadania jest taki, że mimo, iż zmniejszamy coraz bardzie $h$ to w pewnym momencie zaczynamy się oddalać od prawdziwej wartości. Wydawałoby się ze zmniejszanie tak $h$ może nas przybliżać jednak przez ograniecznia arytmetyki ucinamy część mantysy $h$ co oddala nas od wyników prawidłowych.
\end{document}